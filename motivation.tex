\documentclass{article}

\title{Motivation letter}
\author{Serge Kozlukov\\
newkozlukov@gmail.com}

\usepackage{biblatex}

\addbibresource{main.bib}

\begin{document}
\maketitle

To admissions boards of Skoltech and HSE:

My name's Serge Kozlukov, and I'm finishing Bachelor of Science programme in
Applied Mathematics and Computer Science at the Voronezh State University.

Some five or six years ago I'd consider myself a man of practice. I was
completely focused on learning a particular software development stack.
Then I found out sole technology stack isn't enough to do meaningful things.
By 2014 my interests had shifted towards applied mathematics (which I was
completely unfamiliar with at the time), algorithms, and data structures.
Thus I got to the Applied Mathematics and Computer Science programme, Voronezh
SU. In the very first year there I realized Math was something much more than
I'd thought it is. It turned out it was \emph{the} culture. I felt in love with
it, and eventually yet inevitably the subject of functional analysis substituted
that of algorithms. Led by Dr. Prof. Anatoly Baskakov, whom I admire so much and
am very thankful to,
I've written and published two articles in peer-reviewed journals (VAK, Scopus)
on the problem of estimating spectrum and eigenvectors of perturbed matrices of
special form~\cite{Koz17,Koz18}.

Now I feel the need to finally achieve a comporomise between theory and
practice. I want to solve problems on the edge of pure mathematics and
real-world applications. One of the fields that might give me such an opportunity
is that of analysis of data, learning from data, and building algorithms that
adjust for data. Basically the problems in this field are these of
probability theory, statistics, and optimal control. It's math with real-world
applications and relatively cheap experiments --- quite convenient!
Also these are actively researched topics right now and they're the
topics that are likely to shape our future --- the future of eliminated
non-intellectual labour, the future of universal basic income, and many other
things that now seem so distant.

The \textbf{Statistical Learning Theory} programme concerns all of the mentioned
topics and, moreover, the programme states exactly the right questions about the
methods currently adopted in the field: Do these methods \emph{really} work the
way we think they do? Why do they work? Under what exact assumptions? Is it the
best we can do?

I can't think of a better field to combine Math and Application,
nor can I think of a better place to get started in this field, than SLT
programme of Skoltech and HSE. I expect to find there knowledgeable and
experienced people to guide me through process, as well as many like-minded
students, who will form the environment of both a healthy competition and
cooperation. On the other hand my background in algorithms and mathematics shall
help me manage the challenges and requirements of the programme.

\printbibliography

\end{document}