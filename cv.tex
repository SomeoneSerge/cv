\documentclass[10pt,a4paper]{moderncv}


\usepackage[margin=1.5cm]{geometry}
\usepackage[inline]{enumitem}
\usepackage{multicol}
\usepackage{fontspec}
\setmainfont[
  Ligatures=TeX,
  Extension=.otf,
  BoldFont=cmunbx,
  ItalicFont=cmunti,
  BoldItalicFont=cmunbi,
]{cmunrm}
\usepackage[citestyle=authoryear,style=alphabetic,natbib]{biblatex}
\addbibresource{main.bib}

\usepackage[firstyear=2014,lastyear=2028]{moderntimeline}
\usepackage{fontawesome5}

\usepackage{url}

\moderncvstyle{classic} 
\moderncvcolor{orange}

\firstname{Sergei}
\lastname{K}
\title{data, formal mathematical models, numerical algorithms, reproducible deployments}
\address{B310, 2, Konemiehentie, 02150 Espoo, Finland}{}{}
\email{sergei DOT kozlukov AT aalto DOT fi}
\extrainfo{%
    \href{https://matrix.to/\#/@ss:someonex.net}{@ss:someonex.net}~%
    \url{https://someonex.net}\\
    \href{https://github.com/SomeoneSerge/}{\faGithub}~%
    \href{https://twitter.com/SomeoneSerge}{\faTwitter}~%
    \texttt{@SomeoneSerge}
    \href{https://mastodon.acm.org/@nobody}{\faMastodon~@nobody@mastodon.acm.org}
}
\photo[40pt]{pic.jpg}

\begin{document}
\maketitle

\section{Previous work}
\tlcventry{2021/03}{2024}{\href{https://research.aalto.fi/en/persons/sergei-kozlukov}{Doctoral Research}}{Visual Computing Group at the Aalto University}{}{}{
	Working on a PhD thesis concerned with inverse problems in graphics,
	computer vision, image matching. In this work I prioritize insights that
	could only be obtained with the use of problem-specific interactive
	visualization and inspection tools developed ad hoc.
}
\tlcventry{2022/03}{0}{\href{https://nixos.org/community/teams/cuda}{Package maintenance}}{\httplink[github.com/NixOS/nixpkgs]{https://github.com/NixOS/nixpkgs/}}{}{}{
	Working on \httplink[CUDA]{https://nixos.org/manual/nixpkgs/unstable/\#cuda} and
	HPC support in~\href{https://github.com/NixOS/nixpkgs/}{nixpkgs}, on a
	voluntary basis. I maintain~\cite{someone-repology} a number of packages,
	contribute and review changes~\cite{someone-nixpkgs-prs,someone-nixpkgs-reviews} relevant to my work,
	and run an out-of-tree CI~\cite{someone-nixpkgs-cuda-ci} for the scientific
	computing packages that rely on ``unfree'' dependencies. I am a member of
	the NixOS' CUDA maintenance team~\cite{nixpkgs-cuda-team}.
}
\tlcventry{2020}{2021/3}{Consulting}{DM for contacts and details}{Moscow}{}{
	Working on a per-project basis for a variety of employers and teams:
	\begin{itemize*}[before={\@},itemjoin={\quad}]
		\item Implementing and tuning custom routines for low-latency (streaming video) image-based camera localization.
		\item Overlooking progress in bootstrapping domain-specific ``AI'' models.
		\item Overlooking a project concerned with large-scale structure from motion and image-based camera localization.
		\item Maintaining and extending basic CRUD applications, testing and refactoring a previously untested legacy codebase.
	\end{itemize*}
}
\tlcventry{2019/7}{2020/2}{"Fullstack"}{\citet{in3D.io},~DM for contacts and details}{Moscow}{}{
	A part-time job involving the ``any-key'' kind of tasks typical of a startup:
	demo apps, experiments, experiment infrastructure, dataset collection, CRUD.
}
\tlcventry{2018}{2020}{MSc in Computer Science}{Skolkovo Institute of Science and Technology, Higher School of Economics}%
{Moscow}{}{
	Worked on and defended my Master's thesis,
	``\href{https://www.hse.ru/en/edu/vkr/368168926}{Geometric Deep
		Learning for Inverse Graphics}''~\citep{msThesis}.
	Edited version available on~\href{https://github.com/SomeoneSerge/ms-thesis/releases}{GitHub}.
	Cf.~also~\cite{geoopt}.
}
\tlcventry{2018/4}{2018/8}{Consulting}{\citet{r3ds}}{Vorone\'z}{}{
	Implementation of numerical algorithms (numerical optimization, geometry processing) and demo apps.
}
\tlcventry{2014}{2018}{BSc in Applied Mathematics}{Vorone\'z State University}{Vorone\'z}{%
}{
	Worked on and defended my Bachelor's thesis, supervised by
	\href{www.mathnet.ru/eng/person8559}{Anatoly Grigorievi\'c Baskakov}, and
	largely building upon his~\citep{baskakov1987theorem}.
	The original thesis is only available in Russian, but it is mostly a compilation of the previously published results~\cite{someone-jpcs-2017,someone-volsu-2017,someone-vspu-2016}.
}
\section{Speaking and writing}
\tllabelcventry{2021}{2025}{2021--2025 (DSc period)}{}{}{}{}{
	Spoke at NixCon'23~\cite{someone-nixcon23}.
}
\tllabelcventry{2018}{2020}{2018--2020 (MSc period)}{}{}{}{}{
	Co-authored and presented~\cite{someone-hse-manopt} the \citet{geoopt}.
}
\tllabelcventry{2014}{2018}{2014--2018 (BSc period)}{}{}{}{}{
	Published in peer-reviewed venues and presented~\citep{someone-jpcs-2017,someone-volsu-2017,someone-vspu-2016,someone-currentp-problems-2017}
}

\newpage
\section{Supplementary}

\subsection{Peer-reviewed publications}
\printbibliography[keyword=peerReviewed,heading=none]

\subsection{Talks, conferences}
\printbibliography[keyword=talk,heading=none]

\subsection*{External references}
\printbibliography[heading=none,notkeyword=my]
\end{document}
