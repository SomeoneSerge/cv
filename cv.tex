\documentclass[10pt,a4paper]{moderncv}


\usepackage[margin=1.5cm]{geometry}
\usepackage[inline]{enumitem}
\usepackage{multicol}
\usepackage{fontspec}
\setmainfont[
  Ligatures=TeX,
  Extension=.otf,
  BoldFont=cmunbx,
  ItalicFont=cmunti,
  BoldItalicFont=cmunbi,
]{cmunrm}
\usepackage[citestyle=authoryear,natbib]{biblatex}
\addbibresource{main.bib}

\usepackage[firstyear=2009,lastyear=2025]{moderntimeline}
\usepackage{fontawesome}

\usepackage{url}

\moderncvstyle{classic} 
\moderncvcolor{orange}

\firstname{Serge}
\lastname{Kozlukov}
\title{Applied mathematics research}
\address{Office B310, Konemiehentie 2, 02150 Espoo, Finland}{}{}
\email{sergei DOT kozlukov AT aalto DOT fi}
\extrainfo{
\url{https://someonex.net}\\
\faGithub~\faTwitter~\faPaperPlaneO: \texttt{@SomeoneSerge}
}
\photo[40pt]{pic.jpg}

\begin{document}
\maketitle

\section{Industry}
\tlcventry{2021/03}{2024}{Working on the PhD thesis}{Visual Computing Group at Aalto}{}{}{
    Since the Spring of 2021 I'm de facto carrying out doctoral studies under
    supervision of Jaakko Lehtinen. The general subject of the studies are
    inverse problems in computer graphics: instead of capturing the appearance
    of things, I seek to capture a representation that would allow reproducing
    that appearance by simulation of physics.
}
\tlcventry{2020/9}{2021/03}{Consulting}{}{}{}{
    Large-scale structure from motion and localization from images
}
\tlcventry{2020/10}{2021/2}{Consulting}{}{}{}{
    Numerical optimization, realtime camera localization based on custom primitives
    \emph{Meta}:
    I'll make an effort to never ever use Bazel again.
}
\tlcventry{2020/7}{2020/9}{Backend}{Freelance}{Moscow}{}{
    \emph{Good}:
    Recovered a legacy outsource backend project from chaotic dependencies
    (basic dependency injection and mocking, brewn on plain functions), and
    other common \emph{Flask}'s fallacies and pythonic abominations (like
    threadlocals-hell or fragile ad hoc magic instead of structured parsing and
    validation with proper error signaling)
    \emph{Meta}:
    Couldn't help getting distracted by the research projects I was carrying out simultaneously
}
\tlcventry{2019/7}{2020/2}{Fullstack R\&D}{\citet{in3D.io}}{Moscow}{}{
    \begin{flushleft}
    \emph{Good}:
    Avatars: my first project involving volumetric data, camera and lighting models, and such.
    I took some part in deployment of the early demos, in setting up a minimal
        dev-ops, in running experiments and writing some little pieces of the
        pipeline, and even in collecting the data.
    \end{flushleft}
    \begin{flushleft}
    \emph{Challenging}:
        First time in a newly formed team without any established processes. We
        had tried many configurations: from pure anarchy, to variations of
        \emph{agile}, to daily standups. That was... a valuable experience.
    \end{flushleft}
    \begin{flushleft}
    \emph{Meta}:
        Building trust is a process. It's tempting to
        give up sleep and try to meet a deadline you already see you're
        failing, but it is better to inform the team in advance about a
        failure. It's tempting to say ``I'll just do it my way and let people
        face the fact, because it's easier done than explained'', but no idea
        can go too far unless it's picked up by the  team and grown into a
        collective effort. Oh, by the way, it sounds only natural that one doesn't
        refactor a piece of code before writing tests for it -- but even such
        obvious rules don't apply when the communication has failed...
    \end{flushleft}
}
\tlcventry{2019/6}{2019/8}{Consulting}{Accounts chamber}{Moscow}{}{
    A gift from Skoltech (that is, a mandatory internship).
    \href{https://github.com/SomeoneSerge/PyVPO1}{Parsing PDFs, linking raw
    data from different sources}, going around toxic people and fanatics.
    \emph{Meta}:
    I always knew that anything touched by a government is a-priori bad. Now
    I've experimentally confirmed the hypothesis and seek no further
    validation. Also, I'm not a fan of unpaid labour and blackmailing.
}
\tlcventry{2018/4}{2018/8}{R\&D}{\citet{r3ds}}{Vorone\'z}{}{
    \begin{flushleft}
    \emph{Good}:
        The work mostly amounted to writing tools to check out ideas (...and I
        was supposed to do it quick -- the constraint that I'm not sure I've
        entirely satisfied), all related, roughly, to geometry processing. I
        implemented my own
        \emph{B-splines}~\citep{de1978practical,de2012splinefunktionen},
        played with template-based \emph{face tracking}, and
        \href{https://www.dgp.toronto.edu/~karan/pdf/ksinghpaperwire.pdf}{deformation
        models}. I had some quite extensive discussions of software engineering
        practices and design with the team, and I sometimes like to think that
        perhaps I too made a small difference.
    \end{flushleft}
    %
    \begin{flushleft}
    \emph{Meta}:
        I've confirmed I actually tend to over-engineer things. Over-engineered
        things are expensive to maintain. On the other hand, I am by this day
        inspired by GNU Eigen's use of meta-programming, doing the bulk of the
        work at compile time. I'm pretty certain that designing domain-specific
        languages to make specific complex problems look simpler is exactly our
        bright future. Not to imply that Eigen is too simple or templates are
        the perfect tool for the job...
    \end{flushleft}
}
\section{Academia}
\tlcventry{2018}{2020}{Master's thesis}{Skolkovo Institute of Science and Technology, Higher School of Economics}%
{Moscow}{}{
    My Master's thesis,
    ``\href{https://github.com/newkozlukov/ms-thesis/releases}{Geometric Deep
    Learning for Inverse Graphics}'', defended on June 8, is available
    online~\citep{msThesis}.
}
\tldatecventry{2019}{
    Co-author,~\citep{geoopt}}{}{}{}{
    \href{https://arxiv.org/abs/2005.02819}{(ICML GRL+ Workshop; previous
    version presented on ELLIS GRDL Workshop)}
    As with my other collaborations
    with Max,~Rasoul,~\&co, I worked out the mathematical grounds and carried out the
    discussions of the software design.
}
\tldatecventry{2019}{
    Co-author,~\citep{denoiseBoth}}{}{}{}{
    \href{https://arxiv.org/abs/2009.04776}
    A very minor, neglegible, part in the implementation (something to do with
    undistortion of pictures taken with a lense)}
\tldatecventry{2018}{\href{https://github.com/newkozlukov/jpcs-2017}{Spectral
properties of perturbations of Kronecker products of self-adjoint matrices with
simple spectrum~\citep{Koz18}}}{}{}{}{
    I then generalized the result to Kronecker prodcuts of matrices, for
    simplicity only considering self-adjoint cases. The intenion was to proceed
    with a general (infinite-dimensional) Banach-space setting, but I never got
    around to actually carry that out.
\parbox{.7\linewidth}{}}
\tldatecventry{2017}{
    \href{https://github.com/newkozlukov/current-problems-2017/blob/master/current-problems-2017.tex}{Spectral
    properties of perturbations of the
    matrix-of-all-ones~\citep{Koz17}}}{}{}{}{
    Having learned about the method of similar operators, I applied it to a toy
    problem: characterizing the spectra and the eigenvectors of perturbations
    of full-ones matrices. I gained more familiarity with the fix-point theory,
    the majorant series techniques, and, genereally, the perturbation analysis.
    This also resulted in my first peer-reviewed publication
}

\tlcventry{2014}{2018}{Bachelor's thesis}{Vorone\'z State University}{Vorone\'z}{%
}{\begin{flushleft}
    Bachelor's degree, supervised by \href{www.mathnet.ru/eng/person8559}{Anatoly Grigorievi\'c Baskakov}.
    %
    A delve into spectral perturbation theory~\citep{baskakov1987theorem},
    ``linear functional analysis'', differential equations, etc.
    In the courses we'd talk a lot about solutions of differential equations
    and optimal control problems without actually solving any.
    \href{https://github.com/newkozlukov/jpcs-2017}{Published my first
    single-author peer-reviewed papers} in mathematical magazines.
    I focused on Mathematics for my Bachelor's, but in spare time I had some
    fun trying to make it in competitive programming (rather unsuccessfully)
    and further messing with mildly technical stuff
    \end{flushleft}
}
\tlcventry{2009}{2012}{Extracurricular activities}{}{Vorone\'z}{}{
    I didn't like school, so I found something better. First acquiantance with
    C\# and the \texttt{.net} stack, basic datastructures, the (wrong, as Uncle
    Bob would say) MVC architecture, SQL, ORMs, sockets and NATs, cryptographic
    primitives, silly carricatures of gamedev (XNA), GUI programming (forms,
    WPF, web frontend) -- learning ``stuff'', here and there. My first
    experience of almost failing the defense by perpetually postponing
    decisions and assembling things overnight. Supervised by Dmitry
    Babi\'c.
    \begin{flushleft}
    \emph{Meta}: I had my repentance early during my Bachelor's. Now I'm not
        touching any Microsoft-tied tools and products. Amen
    \end{flushleft}
}

\newpage
\section{Supplementary}

\subsection{Peer-reviewed publications}
\printbibliography[keyword=peerReviewed,heading=none]

\subsection{Talks, conferences}
\nocite{*}
\printbibliography[keyword=talk,heading=none]

\subsection*{External references}
\printbibliography[heading=none,notkeyword=my]
\end{document}
