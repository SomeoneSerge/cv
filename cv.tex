\documentclass[10pt,a4paper]{moderncv}


\usepackage[margin=1.5cm]{geometry}
\usepackage[inline]{enumitem}
\usepackage{multicol}
\usepackage{fontspec}
\setmainfont[
  Ligatures=TeX,
  Extension=.otf,
  BoldFont=cmunbx,
  ItalicFont=cmunti,
  BoldItalicFont=cmunbi,
]{cmunrm}
\usepackage[citestyle=authoryear,style=alphabetic,natbib]{biblatex}
\addbibresource{main.bib}

\usepackage[firstyear=2014,lastyear=2028]{moderntimeline}
\usepackage{fontawesome5}

\usepackage{url}

\moderncvstyle{classic} 
\moderncvcolor{orange}

\firstname{SomeoneSerge}
\lastname{}
\title{data-driven algorithms, mathematical modeling}
% \address{...}{}{}
\email{else@someonex.net}
\extrainfo{%
    \httplink[@ss:someonex.net]{matrix.to/\#/@ss:someonex.net}~%
    \url{https://someonex.net}\\
    \httplink[{\faGithub}~\texttt{@SomeoneSerge}]{github.com/SomeoneSerge/}
    \httplink[\faOrcid]{orcid.org/0000-0002-4951-4497}~%
    \httplink[\faIcon{graduation-cap}]{scholar.google.com/citations?user=OCo1DVYAAAAJ}
    \httplink[\faMastodon~@nobody@mastodon.acm.org]{mastodon.acm.org/@nobody}
}
\photo[42pt]{pic.jpg}

\begin{document}
\maketitle

\section{Intro}

SomeoneSerge trades in making complex systems simpler.

\section{Previous work}
\tlcventry{2024/10}{0}{Consulting as a sole proprietor~\faIcon{handshake}}{\httplink[tmi.someonex.net]{tmi.someonex.net}}{Helsinki}{}{
	I offer consulting services on per-project and hourly bases.
}
\tlcventry{2022/03}{0}{\httplink[Nixpkgs duties]{nixos.org/community/teams/cuda}}{\httplink[github.com/NixOS/nixpkgs~\faSnowflake]{github.com/NixOS/nixpkgs/}}{}{}{
	Nixpkgs is an ὰγορα or meydan-like space where people collectively decide how pieces of potentially conflicting computer software may be reliably put together.
	It is the executable ``Wikipedia''.
	I consume Nixpkgs for most of my enterprises, and I participate in Nixpkgs' life.~\cite{someone-repology,someone-nixpkgs-prs,someone-nixpkgs-cuda-ci,nixpkgs-cuda-team,someone-nixos-discourse,nix-dev-dialogues-23,ftn24}
	Examples of what I worked with include: CUDA and ROCm, SLURM and MPI, containers, python3Packages, setup-hooks, CMake, cc-wrapper, dynamic loading, cross-compilation, ``AI and LLM tools''.
	I broke and fixed things.
	Mentored a GSoC project for the NixOS Foundation.~\cite{evanixReport,evanixProposal,evanix}.
	Helped to organize a local Nix user group in Helsinki~\cite{nixInHelsinki}.
}
\tlcventry{2019/7}{2021/3}{Consulting as self-employed~\faIcon{handshake}}{DM for contacts and details}{Moscow}{}{
	I offer consulting services on the project and hourly bases. Examples of what I've worked on:
	\begin{itemize*}[before={\@},itemjoin={\quad}]
		\item Image-based \faCamera~camera localization and SfM
		\item Custom routines for low-latency (streaming video) numerical optimization
		\item Automatic differentiation
		\item Bootstrapping domain-specific ``AI''~models (e.g.\ segmentation)
		\item Basic CRUD, web, refactoring and testing
		\item Legacy codebases
		\item Human avatar reconstruction
		\item \faRunning~``Any-key'' work: demos, experiments, infrastructure, dataset collection
	\end{itemize*}
}
\tlcventry{2018/4}{2018/8}{RnD}{\citet{r3ds}}{Vorone\'z}{}{
	Numerical algorithms, \faCube~geometry processing for VFX
}
\section{Academic record}

I've been walking on and off the academic track, in search of a venue where people focus on exploring the unknown, solve problems the principled way, and take bigger risks (Academia may not necessarily be any of that).

\tlcventry{2021/03}{0}{\href{https://research.aalto.fi/en/persons/sergei-kozlukov}{Doctoral Research~\faIcon{user-graduate}}}{Visual Computing Group, Aalto U, Espoo, Finland}{}{}{
	I work at the Aalto University~\citep{erc-pipelehtinen}
	on a PhD thesis aspiring to democratize, in terms of cost, complexity, and
	reliability, methods and tools for ``computer vision'' and, generally,
	inverse graphics problems.
}
\tlcventry{2018}{2020}{MSc in Computer Science (Statistical Learning Theory)~\faIcon{user-graduate}}{Skolkovo Institute of Science and Technology, Higher School of Economics}%
{Moscow}{}{
	Master's thesis:
	``\href{https://www.hse.ru/en/edu/vkr/368168926}{Geometric Deep
		Learning for Inverse Graphics}''~\citep{msThesis}.
	Edited version hosted on~\href{https://github.com/SomeoneSerge/ms-thesis/releases}{GitHub}.
	Attended a broad selection of courses and seminars such as that of the~\httplink[BayesGroup]{bayesgroup.org}.
}
\tlcventry{2014}{2018}{BSc in Applied Mathematics~\faIcon{user-graduate}}{Vorone\'z State University}{Vorone\'z}{%
}{
	Bachelor's thesis based on peer-reviewed publications~\cite{someone-jpcs-2017,someone-volsu-2017,someone-vspu-2016}, supervised by
	\href{www.mathnet.ru/eng/person8559}{Anatoly Grigorievi\'c Baskakov},
	largely building upon his~\citep{baskakov1987theorem}.
	Participated in the group's special seminar, ``spectral theory of linear operators''.
}

\section{Speaking, writing, public presence}
\tllabelcventry{2021}{0}{2021--2026 (DSc period)}{}{}{}{}{
	\httplink[NixCamp'24]{https://nix.camp/},
	\httplink[Nix in Helsinki]{https://nix-fi.github.io},
	\httplink[Full-Time Nix Podcast]{https://fulltimenix.com/episodes/someoneserge}~\citep{ftn24},
	\httplink[Tropical Probabilistic AI'24]{https://tropical.probabilistic.ai/program/},
	\httplink[2023 Nix Developer Dialogues]{https://discourse.nixos.org/t/2023-nix-developer-dialogues-live-stream/35386}~\citep{nix-dev-dialogues-23},
	\httplink[OceanSprint'23]{https://oceansprint.org/},
	NixCon'23~\faIcon[regular]{snowflake}~\cite{someone-nixcon23},
    \httplink[Geometry and Machine Learning]{gaml.mathi.uni-heidelberg.de} (remote).
}
\tllabelcventry{2018}{2020}{2018--2020 (MSc period)}{}{}{}{}{
	Co-authored and presented~\cite{someone-hse-manopt} the \citet{geoopt}.
}
\tllabelcventry{2014}{2018}{2014--2018 (BSc period)}{}{}{}{}{
	Published in peer-reviewed venues and presented~\citep{someone-jpcs-2017,someone-volsu-2017,someone-vspu-2016,someone-currentp-problems-2017}
}

\newpage
\section{Supplementary}

\subsection{Peer-reviewed publications}
\printbibliography[keyword=peerReviewed,heading=none]

\subsection{Talks, conferences}
\printbibliography[keyword=talk,heading=none]

\subsection*{External references}
\printbibliography[heading=none,notkeyword=my]
\end{document}
