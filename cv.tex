\documentclass{moderncv}


\usepackage[inline]{enumitem}
\usepackage{fontspec}
\setmainfont[
  Ligatures=TeX,
  Extension=.otf,
  BoldFont=cmunbx,
  ItalicFont=cmunti,
  BoldItalicFont=cmunbi,
]{cmunrm}

\usepackage[natbib=true]{biblatex}
\addbibresource{main.bib}

\moderncvtheme[black]{classic}

\firstname{Sergei}
\lastname{Kozlukov}
\title{Aspiring mathematician}
\address{5, Tsimlyanskaya}{109~559~Moscow}{Russia}
\mobile{+7~(900)~945~5929}
\email{newkozlukov@gmail.com}
\social[github]{newkozlukov}
\social[twitter]{newkozlukov}
\social[linkedin]{newkozlukov}
\extrainfo{https://newkozlukov.gitlab.io/}
\photo[64pt]{pic.jpg}

\begin{document}
\maketitle

\section{Personal}
\cvline{Born}{Nov 14, 1996}
\cvline{Citizenship}{Russia}

\section{Hard skills}
\cvlistitem{Math.}
\cvlistitem{Done some time with
    \texttt{Python},
    \texttt{C},
    \texttt{C++},
    \texttt{Java},
    \texttt{C\#+.NET},
    etc
}
\cvlistitem{
    Standard libraries, build systems, version control, tests, etc.
}
\cvlistitem{Algorithms and data structures
(a bit of ICPC, Sedgewick@coursera, etc).}
\cvlistitem{Networking in \(*\)-nix (\texttt{iproute2}, \texttt{aircrack-ng}, etc).}
\cvlistitem{Bits of \texttt{Kademlia}, \texttt{cjdns}, onion routing (at least, concept-wise), \texttt{Tox}.}
\cvlistitem{Bits of systemd,
very scarce bits of linux kernel familiarity.}
\cvlistitem{Bits of security (LUKS, GPG, PKI, WoT, CoT, OTR, etc).}
\cvlistitem{A tiling WM+\texttt{neovim} kind of person.}
\cvlistitem{Beginning
to learn \texttt{pytorch} stack. I do not yet
have experience of any finished DL project.
However I want to believe that I might've gained over time, as a Mathematician
and a person who has always been curious about design and development,
some related skills that would help me close that gap in a short timespan.}

\section{Soft skills}
\cvlistitem{\textbf{Math}.}
\cvlistitem{Only learning to communicate with people,
but I \emph{do} learn.}
\cvlistitem{Coming from academic environment, I'm probably going to need, at
    the early stages, some mentorship (in terms of splitting a problem into
    small steps, maintaining constant progress, etc), as well as oportunity to
    ask stupid questions, and casual reminders that I shouldn't waste time
seeking for a "perfect" solution.}

\section{Research interests}
\cvlistitem{Linear analysis, perturbation theory, method of similar operators.}
\cvlistitem{B-Splines.}
\cvlistitem{Optimal control.}

\subsection{Subjects I've no strong background in/subjects under study:}
\cvlistitem{Optimal transportation theory (Monge-Kantorovich problems).}
\cvlistitem{\textbf{Geodesics}, Global curvature spaces,
    concentration inequalities, esp. variance inequality;
stability of barycenters.}
\cvlistitem{\textbf{Riemannian optimization}.}
\cvlistitem{\textbf{Geometric methods in deep learning}.}
\cvlistitem{\textbf{Hyperbolic deeplearning}.}
\cvlistitem{Dual methods in optimization, accelerated methods.}
\cvlistitem{Projective dynamics, FEM, heat methods; anything related
to simulation of deformable objects;
anything related to \textbf{VFX}.}
\cvlistitem{Embedded Linux, U-Boot.}
\cvlistitem{Coq.}


\section{Education}
\cventry{2009--2012}{Data processing op.}{}{Voronezh}{}{\textbf{.NET development}, supervised by Dmitry
Babich}
\cventry{2014--2018}{B.~Sc.}{Voronezh State
  University}{Voronezh}{\textit{with honours}}{
  Mechanics department, then Applied Mathematics and
  Informatics program, then \textbf{Nonlinear Dynamics} department. 
  Worked in the field of functional analysis and perturbation theory
  \textbf{supervised by} Dr. Prof. \textbf{Anatoly Baskakov}.
  In particular studied and applied the abstract method of
  similar operators~\cite{baskakov1987theorem}, developed by Anatoly Baskakov,
  to the problems of investigating spectral properties of
  \begin{enumerate*}
      \item
          perturbations of the matrix-of-all-ones~\cite{Koz17},
      \item perturbations of Kronecker products of self-adjoint matrices with simple spectrum~\cite{Koz18}.
  \end{enumerate*}
}
\cventry{2018--present}{M.~Sc.}{Skoltech and Higher School of Economics}%
{Moscow}{}{}{Statistical Learning Theory}

\section{Industry experience}
\cventry{April~2018--Sep~2018}{R\&D}{\citet{r3ds}}{Voronezh}{}{
    \textbf{Splines}, mesh deformation tools, facial emotions reconstruction;
    reproduced results of certain papers, built demos
    (technologies used: \textbf{\texttt{C++}}, \texttt{Qt}, proprietary \texttt{3D}-stack);
    studied works of Carl de Boor's related to interpolation and approximation.
}

\section{Peer-reviewed publications}
\printbibliography[keyword=peerReviewed,heading=none]

\section{Talks, conferences}
\nocite{*}
\printbibliography[keyword=talk,heading=none]

\section{Languages}
\cvlanguage{Russian}{Native}{}
\cvlanguage{English}{Comfortable}{}

\newpage
\section*{External references}
\printbibliography[heading=none,notkeyword=my]
\end{document}
