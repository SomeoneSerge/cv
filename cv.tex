\documentclass[10pt,a4paper]{moderncv}


\usepackage[margin=1.5cm]{geometry}
\usepackage[inline]{enumitem}
\usepackage{multicol}
\usepackage{fontspec}
\setmainfont[
  Ligatures=TeX,
  Extension=.otf,
  BoldFont=cmunbx,
  ItalicFont=cmunti,
  BoldItalicFont=cmunbi,
]{cmunrm}
\usepackage[citestyle=authoryear,natbib]{biblatex}
\addbibresource{main.bib}

\usepackage[firstyear=2009,lastyear=2020]{moderntimeline}
\usepackage{fontawesome}

\usepackage{url}

\moderncvstyle{classic} 
\moderncvcolor{orange}

\firstname{Serge}
\lastname{Kozlukov}
\title{Researcher, SWE}
\address{Ziferblat, 19c1, Kuznetsky Most, 107031, Moscow}{}{}
\mobile{+7~(900)~945~5929}
\email{newkozlukov@gmail.com}
% \email{Sergei.Kozlukov@skoltech.ru}
% \email{svkozlukov@edu.hse.ru}
\extrainfo{
\url{https://newkozlukov.gitlab.io/}\\
\faGithub~\faTwitter~\faLinkedin~\faPaperPlaneO: \texttt{@newkozlukov}
}
%\photo[64pt]{pic.jpg}

\begin{document}
\maketitle

Current intersts include hyperbolic deep learning and inverse graphics.
I'm also going to need time that I can invest in learning categories and types,
and catching up with all sorts of LISPs.

\section{Industry}
\tlcventry{2020/7}{0}{Backend}{some nameless startup}{Moscow}{}{
    Taking some time off Academia and doing code-monkey's job.
    Helps with anxiety and covers the rent.
}
\tlcventry{2019/7}{2020/2}{R\&D}{\citet{in3D.io}}{Moscow}{}{
    Worked with kinect/dynamic fusion-like algorithms,
    camera calibration and undistortion.
    (Too) actively maintained back-end
    (docker, multiple GPUs, redis, segmentation/detection/etc-as-a-service).
    Took part in introducing DevOps practices.
    Took RGBD recordings of some dozens of humans.
    In3D has been an immensely important experience for me, as
    we've been building the team and the company from scratch, without any such
    prior experience. We've tried different many configurations, from total
    anarchy, to daily reports, to offline daily standups, to mandatory offline
    presence -- collectively searching for the team's optimum.
}
\tlcventry{2018/4}{2018/8}{R\&D}{\citet{r3ds}}{Vorone\'z}{}{
    Tracking, rigging, mesh-deformation, etc-research,
    and software engineering.
    Wrote a small splines library and patch-based face tracker, both
    admittedly over-engineered.
    (Timidly) I also may have introduced the team to some new SWE practices and
    a little different perspective on agile, but I'm not certain.
    Tech stack included post-11 C++, Qt, qmake, GNU Eigen.
    This internship was limited in time because of upcoming Master's.
    During that time I played a little with C++'s type system,
    learned about VFX industry, learned basics of the rendering
    pipelines, and also discovered the great Carl de Boor~\citep{de1978practical,de2012splinefunktionen},
    inventor of B-Splines, student of Isaac ``the father of splines'' Schoenberg,
    who proved to me once again that applied problems
    can be expressed in a clean functional-analytic manner
    and that it is beneficial.
}
\section{Academia}
\tldatecventry{2019}{
    Co-author,~\citep{geoopt}.}{}{}{}{
    (ICML Workshop (GRL+), previous version presented on ELLIS GRDL Workshop)
    Presenting the Geoopt package for Riemannian optimization in PyTorch, but
    also trying to demonstrate an ideology, considering optimization process
    from both analytic (differential-geometric) and synthetic
    (metric-geometric) points of view.
}
\tldatecventry{2019}{
    Co-author,~\citep{denoiseBoth}}{}{}{}{
    (Submitted to 3DV, under review)
    Overall, the paper is concerned with a semi-supervised procedure
    for denoising the output of depth sensors.
    My role was on the implementation side, related to calibration of sensors.
}
\tlcventry{2018}{2020}{MS in Statistical Learning Theory}{Skoltech and Higher School of Economics}%
{Moscow}{}{
    On June 8, 2020, I have defended my Master's thesis!
    The script ``Geometric Deep Learning for Inverse graphics'', is available
    online~\citep{msThesis}.
    My Master's degree was some rock-opera sort of story:
    I have changed three thesis topics and two supervisors, not counting the
    formal ones. I had (the autumn of 2018) intended to work on optimal
    transportation with Thibaut Le Gouic, but soon switched to researching
    hyperbolic neural networks (late winter of 2019),
    without supervision, collaborating with my peers, Max Kochurov and Rasoul Karimov.
    Finally, after we had decided our results weren't good enough for NeurIPS,
    I had a long break (July 2019--February 2020) working on differentiable
    rendering and fusion-like algorithms, in cooperation with Dmitry
    Ulyanov. Master's degree was an important opportunity of exploration and
    self-search. Now that it's done, I see the Master's as a short intermediate
    stage in the Academic path, which allows one to probe several directions
    and determine the actual interests before beginning a full-blown PhD.
    Aside from catching up on differential geometry, I have learned a lot about
    communication, collaboration, failure, and about positioning myself and my
    research. Master's also has softened me, somewhat.
}
\tldatecventry{2018}{Spectral properties of perturbations of Kronecker products of self-adjoint matrices with simple spectrum~\citep{Koz18}}{}{}{}{
    I then generalized the result to Kronecker prodcuts of matrices, for simplicity
    only considering self-adjoint cases.
\parbox{.7\linewidth}{}}
\tldatecventry{2017}{
    Spectral properties of perturbations of the matrix-of-all-ones~\citep{Koz17}}{}{}{}{
    After learning about the method of similar operators,
    I applied it to a toy problem of analyzing
    perturbations of all-ones matrix,
    gaining more familiarity with fixpoint theory
    and majorant series techniques in perturbation analysis.
    This resulted in my first peer-reviewed publication
}

\tlcventry{2014}{2018}{BS in Applied Mathematics}{Vorone\'z State University}{Vorone\'z}{%
}{\begin{flushleft}
    Perturbation theory, linear functional analysis, ordinary differential
    equations, optimal control, etc.
    %
    Here I started my acquiatance with Mathematics, which to a significant
    extent is due to (my then-to-be supervisor) Anatoly Baskakov and his
    students (Yegor Dikarev, Valery Kharitonov, ...).
    Their group's introductory summer lecture was my first exposure to the
    ``true mathematics'', after which my second period of self-education
    began.
    I initially got to the Mechanics department, and even took
    two out of three ``Theoretical Mechanics'' courses.
    However, the mathematical curriculum in that department was
    hopelessly outdated and I was learning everything by myself
    from textbooks.
    The second year, I transitioned to the ``Applied Mathematics and
    Informatics'' (read CS) department,
    spending half of the year on fulfilling academic difference,
    in spare attempting to organize our ICPC team.
    %
    That year I acquianted with Anatoly Baskakov,
    during his ``functional analysis'' course
    (which in fact was a course about just everything)
    and research seminar.
    %
    By the third year my priorities'd completely shifted from competitive
    programming to mathematics.
    Supervised by Anatoly Grigorievi\'c (now in ``nonlinear dynamics'' dpt.)
    I'd been spending time on graphs of integral spectra, until I came across
    his ``method of similar operators''~\citep{baskakov1987theorem},
    related to estimation of spectral properties of perturbed operators.
    This work has resulted in several peer-reviewed articles and talks.
    %
    I had several internships during my BS, all of them (as well as ICPC
    attempts) have pointed out difficulties in communication (read
    ``teamwork''), which I worked intensively on later.
    This period was also associated with certain disappointment regarding
    software engineering world (base and servile), until the very end of BS,
    when I started worrying about inapplicability of my mathematics research to
    real world (which turned me to VFX and then to DL).
    \end{flushleft}}

\tlcventry{2009}{2012}{``Data Processing in .NET'' course}{}{Vorone\'z}{}{
    A rather generic course on C\#/dotnet stack, under supervision of Dmitry Babi\'c.
    This period was more about active independent exploration of all things
    related (from networking, to SQL, to Web, to 3D, to design patterns) than
    about the course itself.  This period provided me with basic acquiatance of
    technical side of SWE world, but less so with the teamwork, communication,
    and disciplines.
    It has also led me eventually to realize that ``\textbf{I need math}'',
    even out of purely applied and servile motives!
}

\newpage
\section{Supplementary}

\subsection{Peer-reviewed publications}
\printbibliography[keyword=peerReviewed,heading=none]

\subsection{Talks, conferences}
\nocite{*}
\printbibliography[keyword=talk,heading=none]

\subsection*{External references}
\printbibliography[heading=none,notkeyword=my]
\end{document}
