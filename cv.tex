\documentclass[10pt,a4paper]{moderncv}


\usepackage[margin=1.5cm]{geometry}
\usepackage[inline]{enumitem}
\usepackage{multicol}
\usepackage{fontspec}
\setmainfont[
  Ligatures=TeX,
  Extension=.otf,
  BoldFont=cmunbx,
  ItalicFont=cmunti,
  BoldItalicFont=cmunbi,
]{cmunrm}
\usepackage[citestyle=authoryear,natbib]{biblatex}
\addbibresource{main.bib}

\usepackage[firstyear=2009,lastyear=2020]{moderntimeline}
\usepackage{fontawesome}

\usepackage{url}

\moderncvstyle{classic} 
\moderncvcolor{black}  

\firstname{Sergei}
\lastname{Kozlukov}
\title{Researcher at in3D.io,
MSc-2 in Statistical Learning Theory at Skoltech and Higher School of Economics}
\address{Ziferblat, 19c1, Kuznetsky Most, 107031, Moscow}{}{}
\mobile{+7~(900)~945~5929}
\email{newkozlukov@gmail.com}
% \email{Sergei.Kozlukov@skoltech.ru}
% \email{svkozlukov@edu.hse.ru}
\extrainfo{
\url{https://newkozlukov.gitlab.io/}\\
\faGithub~\faTwitter~\faLinkedin~\faPaperPlaneO: \texttt{@newkozlukov}
}
%\photo[64pt]{pic.jpg}

\begin{document}
\maketitle

At the moment, I'm interested in optimization on manifolds,
photorealistic graphics, inverse graphics,
and hyperbolic neural networks.

\section{Industry}
\tlcventry{2019/7}{0}{R\&D}{\citet{in3D.io}}{Moscow}{}{
    CV research, software engineering, microservices, data collection, devops...
    It's a work in active progress
}
\tlcventry{2018/4}{2018/8}{R\&D}{\citet{r3ds}}{Vorone\'z}{}{
    Constrained in time because of upcoming MSc, I spent most of my R3DS time on research,
    reproducing papers and building demos, only production thing being small splines-related library.
    R3DS gave me useful C++ experience, acquiatance with 3D and VFX basics and their open problems.
    I learned about some mesh deformation techniques, patch-based models for facial expressions, and more.
    I learned about Eigen, whose clever use of type system not only for implementing
    both policies, and constraints, and compile-time polymorphism
    gives almost physical satisfaction.
    I also discovered for myself the great Carl de Boor~\citep{de1978practical,de2012splinefunktionen},
    inventor of B-Splines, student of Schoenberg (the ``father of splines'').
    Carl de Boor turned out to be pretty good at expressing applied problems and solutions
    in clean and rigorous functional-analytic form, while his lectures
    on functional analysis and linear algebra, still being rigorous,
    are grounded on computational considerations.
    A great example of combining applications and mathematical rigour.
}
\section{Academia}
\tlcventry{2018}{2020}{M.~Sc. in Statistical Learning Theory}{Skoltech and Higher School of Economics}%
{Moscow}{}{
    I applied for SLT program, aspiring to bring one day rigorous "real" math
    into pretty active applied world of machine learning.
    I started out with visiting optimal transportation seminars
    and passing a lot of obligatory courses on optimization.
    I first wanted to do research with Thibaut Le Gouic, as he was
    very major and honourable figure in optimal transportation world.
    I was to continue his research on relationship
    between variance inequality, and curvature bounds in metric spaces,
    existence of barycentres and extendibility of geodesics.
    However, actively involved in mandatory Skoltech courses,
    I quickly switched to hyperbolic embeddings
    and hyperbolic neural networks, which we worked on for a course project.
    This eventually resulted in new research that is still a work in active progress
    and I hope some day will be published.
    Now I got more interested in photorealism, and, participating in a startup
    on human avatars, I hope to do research and thesis on texturing
    and estimating reflectance information in unconstrained setting
    (non-static objects, unknown and changing lighting, etc).}
\tldatecventry{2018}{Spectral properties of perturbations of Kronecker products of self-adjoint matrices with simple spectrum~\citep{Koz18}}{}{}{}{
    I then generalized the result to Kronecker prodcuts of matrices, for simplicity
    only considering self-adjoint cases.
\parbox{.7\linewidth}{}}
\tldatecventry{2017}{
    Spectral properties of perturbations of the matrix-of-all-ones~\citep{Koz17}}{}{}{}{
    After learning about the method of similar operators,
    I applied it to a toy problem of analyzing
    perturbations of all-ones matrix,
    gaining more familiarity with fixpoint theory
    and majorant series techniques in perturbation analysis.
    This resulted in my first peer-reviewed publication
}

\tlcventry{2014}{2018}{B.~Sc. in Applied Mathematics}{Vorone\'z State University}{Vorone\'z}{%
}{\begin{flushleft}
      Here I started my acquiatance with Mathematics.
      It all began with the introductory lecture
      read by my then-to-be friend Valery Kharitonov
      (who by accidence, had significantly influenced my further path).
      He provided us with ``proper'' set-theoretic definition of a function.
      It was that moment when he proved bijective functions to be
      exactly ones simultaneously injective and surjective,
      that I realized: all qualities of programming that I respected
      are in fact characteristic of Math.
      %
      I initially got to Mechanics department, and even took
      two out of three courses on ``Theoretical Mechanics''.
      But the very next year I switched to department that was more active
      Math- and coding-wise and had better lecturers.
      Half of my second year I spent filling in the academic difference
      (which was eleven additional exams if I remember),
      meanwhile trying to come to terms with our ICPC team.
      %
      It was also the time I acquianted with \textbf{Anatoly Baskakov}, 
      my future supervisor, who was reading a course titled ``functional analysis''.
      It was in fact a course on everything, because Anatoly Grigorievi\'c
      was (just as I) extremely dissatisfied with how Probability Theory
      was read in our university, and how PDEs were read,
      and Complex Calculus, and actually Real Calc too.
      %
      On my third year my priorities completely shifted from competitive programming
      to mathematics. Supervised by Anatoly Grigorievi\'c I spent some time
      on graphs of integral spectra, until I came across his
      ``method of similar operators''~\citep{baskakov1987theorem}
      which basically is a means of constructing a nonlinear equation
      that after application of Banach's fixpoint theorem yields you decent estimates
      on spectral properties (like eigenvalues and eigenfunctions) of perturbed linear operators,
      provided you know the spectral structure of unperturbed ideal operator.
    \end{flushleft}}

\tlcventry{2009}{2012}{A course on ``Data Processing in .NET''}{}{Vorone\'z}{}{
    Passed a rather generic course on CS/dotnet stack, under supervision of Dmitry Babi\'c.
    It was more about active independent exploration of all things related
    (from low-level networking, to web, to 3D graphics)
    than about the course itself.
    This period provided me with basic acquiatance
    of technical side of SWE world, but less so with the teamwork,
    communication, and discipline aspect.
    It also led me eventually to the decision
    that ``\textbf{I need math}''. Purely applied and servile motives!
}

\newpage
\section{Supplementary}

\subsection{Peer-reviewed publications}
\printbibliography[keyword=peerReviewed,heading=none]

\subsection{Talks, conferences}
\nocite{*}
\printbibliography[keyword=talk,heading=none]

\subsection*{External references}
\printbibliography[heading=none,notkeyword=my]
\end{document}
