\documentclass[10pt,a4paper]{moderncv}


\usepackage[margin=1.5cm]{geometry}
\usepackage[inline]{enumitem}
\usepackage{multicol}
\usepackage{fontspec}
\setmainfont[
  Ligatures=TeX,
  Extension=.otf,
  BoldFont=cmunbx,
  ItalicFont=cmunti,
  BoldItalicFont=cmunbi,
]{cmunrm}
\usepackage[citestyle=authoryear,natbib]{biblatex}
\addbibresource{main.bib}

\usepackage[firstyear=2009,lastyear=2020]{moderntimeline}
\usepackage{fontawesome}

\usepackage{url}

\moderncvstyle{classic} 
\moderncvcolor{orange}

\firstname{Serge}
\lastname{Kozlukov}
\title{Researcher, SWE}
\address{Ziferblat, 19c1, Kuznetsky Most, 107031, Moscow}{}{}
\mobile{+7~(900)~945~5929}
\email{newkozlukov@gmail.com}
% \email{Sergei.Kozlukov@skoltech.ru}
% \email{svkozlukov@edu.hse.ru}
\extrainfo{
\url{https://newkozlukov.gitlab.io/}\\
\faGithub~\faTwitter~\faLinkedin~\faPaperPlaneO: \texttt{@newkozlukov}
}
%\photo[64pt]{pic.jpg}

\begin{document}
\maketitle

Hyperbolic DL, Inverse Graphics.

\section{Industry}
\tlcventry{2020/7}{2020/9}{Backend}{Freelance}{Moscow}{}{
    Finished a legacy backend project: introduced means for basic
    \emph{dependency injection} working around \emph{Flask}'s threadlocals-hell to
    allow mocking and integration testing (using
    \href{https://github.com/testcontainers/testcontainers-python}{\emph{testcontainers}}),
    set up organised validation and error signaling, etc.
}
\tlcventry{2019/7}{2020/2}{R\&D}{\citet{in3D.io}}{Moscow}{}{
    \emph{Good}:
    Worked with \emph{volumetric fusion}, camera \emph{calibration}, and
    undistortion; e.g. came with cheap heuristics for RGBD sequence
    \emph{decimation}.  Backend (\emph{docker}, \emph{redis}, multi-GPU,
    model-as-a-service).  Put the beginning to decomposing a 1300LOC
    \emph{pytorch}-mixed-with-IO untest\{ed|able\} core function...  Took part
    in introducing minimal \emph{DevOps} practices.  Collected data outdoors.
    \emph{Challenging}:
    It was my first time in a newly formed team without established processes.
    We'd been trying out many different configurations (from anarchy to
    \emph{agile} and daily standups) and that was a valuable experience.
    \emph{Bad}:
    I have repeatedly failed communication-wise.  I made promises I failed to
    keep, I failed to report in advance the deadlines I was failing, I've
    several times chosen to ``just do it my way without explaining the reasons,
    "because it's easier done than said"'' and of course that wasn't conductive
    to building the \emph{trust}.
}
\tlcventry{2018/4}{2018/8}{R\&D}{\citet{r3ds}}{Vorone\'z}{}{
    \emph{Good}:
    Wrote a small
    \emph{B-splines}~\citep{de1978practical,de2012splinefunktionen} library, a
    patch-based \emph{face tracking} algo, a
    \href{https://www.dgp.toronto.edu/~karan/pdf/ksinghpaperwire.pdf}{wires}-like
    deformations demo.  I may have to some extent impacted team's perspective
    on software design and processes.
    \emph{Stack}: post-11 C++, Qt, qmake, GNU Eigen.
    \emph{Bad}: I definitely had over-engineered many things. I wasn't easy
    in communication. I had a deadlines problem.
}
\section{Academia}
\tlcventry{2018}{2020}{MS in Statistical Learning Theory}{Skoltech and Higher School of Economics}%
{Moscow}{}{
    My Master's thesis,
    ``\href{https://github.com/newkozlukov/ms-thesis/releases}{Geometric Deep
    Learning for Inverse Graphics}'', finally defended on June 8, is available
    online~\citep{msThesis}.  It's been a rock-opera style story of active
    search and a sampling of many different subjects (and people) -- follow the
    link to learn more!
}
\tldatecventry{2019}{
    Co-author,~\citep{geoopt}.}{}{}{}{
    (ICML Workshop (GRL+), previous version presented on ELLIS GRDL Workshop)
    Geoopt is a package for Riemannian optimization and manifolds-aware models
    in Pytorch. While the implementation was largely taken care of by Max Kochurov
    and Rasoul Karimov, I worked out the mathematical grounds and controlled
    the design.
    The workshop paper, besides introducing the package, is also trying to
    demonstrate an ideology of treating the optimization process from both
    analytic (differential-geometric) and synthetic (metric-geometric) points
    of view.
}
\tldatecventry{2019}{
    Co-author,~\citep{denoiseBoth}}{}{}{}{Very minor role on the implementation side}
\tldatecventry{2018}{Spectral properties of perturbations of Kronecker products of self-adjoint matrices with simple spectrum~\citep{Koz18}}{}{}{}{
    I then generalized the result to Kronecker prodcuts of matrices, for simplicity
    only considering self-adjoint cases.
\parbox{.7\linewidth}{}}
\tldatecventry{2017}{
    Spectral properties of perturbations of the matrix-of-all-ones~\citep{Koz17}}{}{}{}{
    After learning about the method of similar operators,
    I applied it to a toy problem of analyzing
    perturbations of all-ones matrix,
    gaining more familiarity with fixpoint theory
    and majorant series techniques in perturbation analysis.
    This resulted in my first peer-reviewed publication
}

\tlcventry{2014}{2018}{BS in Applied Mathematics}{Vorone\'z State University}{Vorone\'z}{%
}{\begin{flushleft}
    Learned about spectral perturbation theory~\citep{baskakov1987theorem},
    linear functional analysis, differential equations, Pontryagin and
    Bellman's principles in control, etc.
    \href{https://github.com/newkozlukov/jpcs-2017}{Published my first
    single-author peer-reviewed papers} in mathematical magazines. Participated
    in ICPC-esque activities.
    %
    Had some software-engineering style internships locally (\emph{Java},
    \emph{Spring}, data stuff in \emph{Python}).
    %
    Defended under supervision of Anatoly Grigorievi\'c Baskakov.
    \end{flushleft}
}
\tlcventry{2009}{2012}{Data Processing in .NET}{}{Vorone\'z}{}{
    That time when I was completely ignoring the school...
    First \emph{real} acquiantance with a programming language (C\#).  Learned
    about \texttt{.net} stack, basic datastructures, (the wrong) MVC
    architecture in web, SQL, ORMs, networking (from sockets to NATs, iptables,
    and hole-punching), assymetric cryptography, silly attempts at gamedev
    (XNA), GUI programming (forms, WPF, web frontend) -- you know, ``stuff'',
    here and there. Under supervision of Dmitry Babi\'c.
}

\newpage
\section{Supplementary}

\subsection{Peer-reviewed publications}
\printbibliography[keyword=peerReviewed,heading=none]

\subsection{Talks, conferences}
\nocite{*}
\printbibliography[keyword=talk,heading=none]

\subsection*{External references}
\printbibliography[heading=none,notkeyword=my]
\end{document}
